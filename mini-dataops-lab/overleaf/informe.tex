\documentclass[11pt,a4paper]{article}
\usepackage[utf8]{inputenc}
\usepackage{graphicx}
\usepackage{hyperref}
\usepackage{caption}
\usepackage{float}
\title{Laboratorio Práctico Clase 6: Implementando un Mini-Pipeline de DataOps}
\author{Tu Nombre\\Curso: Modelos y Bases de Datos}
\date{\today}

\begin{document}
\maketitle
\begin{abstract}
Informe del laboratorio: diseño, implementación y evidencias del mini-pipeline de DataOps usando AWS.
\end{abstract}

\section{Introducción}
Explica objetivos y datasets usados:
\begin{itemize}
  \item SpotifyAudioFeaturesApril2019.csv
  \item Global Music Artists.csv
\end{itemize}

\section{Diseño del pipeline}
Describe el flujo ELT y el stack seleccionado (AWS). Incluye diagrama (puedes pegar una imagen).
% Inserta diagrama aquí si quieres
\begin{figure}[H]
\centering
\framebox{\parbox{0.9\linewidth}{Diagrama del pipeline: Lambda (ingest) -> S3 raw -> Glue (transform) -> S3 processed}}
\caption{Diagrama conceptual del pipeline}
\end{figure}

\section{Implementación}
\subsection{Almacenamiento (S3)}
Explica la estructura de buckets y carpetas. \\
\textbf{Captura 1:} Inserte aquí la captura del Bucket S3 mostrando la carpeta \texttt{raw} y \texttt{processed}.
\begin{figure}[H]
\centering
\includegraphics[width=0.9\linewidth]{s3_buckets.png}
\caption{Buckets S3 (raw y processed).}
\end{figure}

\subsection{Ingesta (Lambda)}
Describe la función Lambda y la programación. \\
\textbf{Captura 2:} Inserte aquí la captura de la consola Lambda con el código y la prueba de ejecución.
\begin{figure}[H]
\centering
\includegraphics[width=0.9\linewidth]{lambda_console.png}
\caption{Lambda function console + CloudWatch log.}
\end{figure}

\subsection{Transformación (Glue)}
Explicar Crawler y Glue Job. \\
\textbf{Captura 3:} Inserte aquí la captura del Glue Job run con status \texttt{SUCCEEDED}.
\begin{figure}[H]
\centering
\includegraphics[width=0.9\linewidth]{glue_job.png}
\caption{Glue Job run.}
\end{figure}

\section{Verificación de calidad de datos}
Describir checks añadidos (p.ej. popularity entre 0 y 100).

\section{Automatización y monitorización}
Describir EventBridge/SNS y CloudWatch para alertas.

\section{Reflexión}
Respuesta a preguntas de la fase 5 (calidad, automatización, monitorización, colaboración).

\section{Anexos}
- Scripts: \texttt{ingest/lambda_function.py}, \texttt{transform/glue_job.py}.
- Checklist de evidencias.
\end{document}
